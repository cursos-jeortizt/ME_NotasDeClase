
%\chapter{TEORIA DE LA UTILIDAD}
%\chapter{MODELOS BAJO INCERTIDUMBRE}
%\chapter{TEORÍA DEL RIESGO Y ÁRBOLES DE DECISIÓN}
%\chapter{ANÁLISIS BAYESIANO}
%\section{}
%\subsection{}
%\chapter{TEORÍA DE JUEGOS}
%\chapter{REDES DE AUTÓMATAS ESTOCÁSTICOS}
%\chapter{DECISIONES MULTIOBJETIVO EN COMUNIDADES DE AGENTES ARTIFICIALES: COOPERACIÓN Y NEGOCIACIÓN}

\chapter{DECISIONES EN COMUNIDADES DE AGENTES ARTIFICIALES}

\section{INTRODUCCIÓN}

\section{PLANTEAMIENTO DEL PROBLEMA}\label{Sec:Planteamiento}

\section{CLASIFICACIÓN DE LOS JUEGOS DE ESTRATEGIA}\label{Sec:Clas_JE}


\section{JUEGOS DE SUMA CONSTANTE ENTRE DOS AGENTES}\label{Sec:Juegos_SC2A}
\subsection{Estrategias dominadas}
\subsection{Solución de un juego de suma constante entre dos agentes}
\subsection{Solución mixta de un juego bipersonal de suma constante}
\subsection{Modelo de programación lineal para juegos de suma constante}

\section{JUEGOS DE SUMA NO CONSTANTE ENTRE DOS AGENTES}\label{Sec:Juegos_SNC2A}
\subsection{Equilibrio puro de Nash de un juego de suma no constante entre dos agentes}
\subsection{Equilibrio de Nash con estrategias mixtas en un juego de suma no constante entre dos agentes}

\section{CASOS DE ESTUDIO}\label{Sec:CE_DCA}
\subsection{Caso 1: Competencia entre dos comunidades}
\subsection{Caso 2: Comunidades de agentes y programación lineal}
