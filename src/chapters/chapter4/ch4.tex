
\chapter{MODELOS MATEMÁTICOS Y SIMULACIÓN}

%Contadores y formato para numeracion de elementos
\renewcommand{\thefootnote}{\arabic{footnote}}
\renewcommand{\theequation}{\arabic{chapter}-\arabic{equation}}
\renewcommand{\thefigure}{\arabic{chapter}.\arabic{figure}}
\renewcommand{\figurename}{Figura}
\renewcommand{\tablename}{\textbf{Tabla}}
\renewcommand{\thetable}{\textbf{\arabic{chapter}-\arabic{table}}}
\newcounter{definitionN}
\newcounter{exampleN}
\newcounter{tableN}
\newcounter{footN}
\newcounter{algorithmN}
\stepcounter{definitionN}
\stepcounter{exampleN}
\stepcounter{tableN}
\stepcounter{footN}
\stepcounter{algorithmN}
\bibliographystyle{apalike}

%Diana Carolina Guarin Angulo

\section{INTRODUCCIÓN}\label{sec:intro}
\setcounter{equation}{0}

\section{VARIABLE TRASCENDENTAL: EL ESPACIO-TIEMPO}\label{sec:varTrasc}

\section{SISTEMA Y AMBIENTE}\label{sec:sistAmbiente}

\section{MODELO DEL SISTEMA}\label{sec:modDelSistema}

\section{SOLUCIÓN DEL MODELO MATEMÁTICO}\label{sec:SolModeloMatematico}
\subsection{Tipos de Soluciones}\label{subsec:TiposDeSoldeModMat}
\subsection{Limitaciones de los Modelos Matemáticos}\label{subsec:LimitdeModMat}

\section{EL MODELO SAO}\label{sec:SAOModel}
\subsection{Módulos del Modelo}\label{subsec:ModulosdelModelo}
\subsection{Especificación Matemática}\label{subsec:EspecifMatem}

\section{SIMULACIÓN DE EVENTOS DISCRETOS}\label{sec:SimulEventDiscret}
\subsection{Arquitectura de la Aplicación }\label{subsec:ArquitDeLaApp}
\subsection{Metodología de la Simulación}\label{subsec:MetodDeLaSimul}

\section{RESUMEN Y PANORAMA HISTÓRICO }\label{sec:ResumenyPanHisoric}

\section{EJERCICIOS}\label{sec:EjerciciosModySimul}


%\chapter{TEOREMA FUNDAMENTAL DE LA SIMULACIÓN}
%\chapter{ARQUITECTURA GENERAL DE UN SIMULADOR DE SISTEMAS COMPLEJOS}
%\chapter{FUNCIONES PERCENTILES,TRUNCADAS Y CONTAMINADAS CONJUNTAS}
%\chapter{MODELOS DE MOVILIDAD}
%\chapter{LENGUAJES DE PROGRAMACIÓN PARA SIMULACIÓN DE REDES}
%\chapter{DISTRIBUCIÓN LAMBDA GENERALIZADA}