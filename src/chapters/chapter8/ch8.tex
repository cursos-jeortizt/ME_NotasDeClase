\chapter{MODELOS MATEMÁTICOS Y SIMULACIÓN}

%Contadores y formato para numeracion de elementos
\renewcommand{\thefootnote}{\arabic{footnote}}
\renewcommand{\theequation}{\arabic{chapter}-\arabic{equation}}
\renewcommand{\thefigure}{\arabic{chapter}.\arabic{figure}}
\renewcommand{\figurename}{Figura}
\renewcommand{\tablename}{\textbf{Tabla}}
\renewcommand{\thetable}{\textbf{\arabic{chapter}-\arabic{table}}}
\stepcounter{definitionN}
\stepcounter{exampleN}
\stepcounter{tableN}
\stepcounter{footN}
\stepcounter{algorithmN}
\providecommand{\abs}[1]{\lvert#1\rvert}
\providecommand{\norm}[1]{\lVert#1\rVert}



\section{INTRODUCCIÓN}\label{sec:intro}
\section{MODELOS DE MOVILIDAD SINTÉTICOS ESTÁNDAR}\label{sec:movilityModels}
\subsection{Random Walk Mobility Model}
\subsection{Random Waypoint}
\subsection{Gauss-Markov}

\section{FAMILIA PARAMÉTRICA DE GAUSS}\label{sec:gaussParamFam}
\subsection{Familia de Gauss Uniparamétrica}
\subsection{Distribución Normal Multivariada}
\subsubsection{Función de densidad}
\subsubsection{Integral de Aitken}
\subsubsection{Función Generadora De Momentos}
\subsubsection{Distribuciones Marginales}
\subsubsection{Distribuciones Condicionales}
\subsubsection{Independencia}

\section{MOVILIDAD GAUSS-MARKOV CON ATRACTORES}
\subsection{Generador de una variable de Gauss Univariada}
\subsection{Generador de un vector dimensional de Gauss}
\subsection{Método de movilidad Gauss-Markov con múltiples atractores}


